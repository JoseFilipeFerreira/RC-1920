\documentclass{llncs}
\usepackage{times}
\usepackage[T1]{fontenc}

\usepackage{a4}
\usepackage[margin=3cm,nohead]{geometry}
\usepackage{epstopdf}
\usepackage{graphicx}
\usepackage{fancyvrb}
\usepackage{amsmath}
\renewcommand{\baselinestretch}{1.5}

\usepackage[utf8]{inputenc}
\usepackage[portuges]{babel}


\begin{document}
\mainmatter
\title{Mobile Networks: From 4G to 5G}

\titlerunning{Paper Title}

\author{João Teixeira \and José Ferreira \and Miguel Solino}

\authorrunning{João Teixeira \and José Ferreira \and Miguel Solino}

\institute{
Universidade do Minho, Departamento de  Informática, 4710-057 
Braga, Portugal\\
e-mail: \{a85504,a83683,a86435\}@alunos.uminho.pt
}

\date{}
\bibliographystyle{splncs}

\maketitle
\begin{abstract}
Resumo...
\end{abstract}

\section{Introdução}

Todas as pessoas sonham puder descarregar ficheiros da Internet a velocidades alucinantes.
Inevitavelmente, esse sonho acabou por sair das nossas casas e
seguir-nos para a rua à medida que os dispositivos se tornavam mais
compactos e mais potentes.
A solução prontamente apresentada pelas empresas foi uma conexão
móvel cada vez mais rápida.
Eventualmente foi atingida uma barreira na procura incessante da
velocidade, o 4G. A resposta das empresas a este entrave foi o
desenvolvimento de um novo paradigma denominado de 5G.

\section{Implementação do 4G}
\section{A barreira do 4G}
\section{Implementação do 5G}
\section{Problemas resolvidos}
\section{Desvantagens inerentes}

According to Table~\ref{tab:TabelaExemplo}...

% Exemplo de uma tabela com duas colunas
\begin{figure}
\centering
\begin{tabular}{|c|c|}\hline
(a) Delay and jiiter & (b) Delay and loss \\ \hline

(c) Delay and throughput & (d) Jitter and loss \\ \hline

(e) Jitter and throughput & (f) Loss and throughput \\ \hline
\end{tabular}
\caption{\label{tab:TabelaExemplo}Tabela exemplo.}
\end{figure}

%\section{Simulation Scenario}

\section{Conclusions}
Neste trabalho...

%UNCOMMENT para a bibliografia 
%% ficheirodebibliografia.bib
%\bibliography{ficheirodebibliografia}

%ou inserir directamente os vários \bibitem 
\begin{thebibliography}{1}
\bibitem{Zadeh65}
Zadeh, L.:
\newblock {Fuzzy sets} (1965)

\bibitem{Nguyen99}
Nguyen, H., Walker, E.:
\newblock {First course in fuzzy logic}.
\newblock {Boca Raton: Chapman and Hall/CRC Press} (1999)
\end{thebibliography}

\end{document}
