\documentclass{llncs}
\usepackage{times}
\usepackage[T1]{fontenc}

\usepackage{a4}
\usepackage[margin=3cm,nohead]{geometry}
\usepackage{epstopdf}
\usepackage{graphicx}
\usepackage{fancyvrb}
\usepackage{amsmath}
\renewcommand{\baselinestretch}{1.5}

\usepackage[utf8]{inputenc}
\usepackage[portuges]{babel}


\begin{document}
\mainmatter
\title{Mobile Networks: From 4G to 5G}

\titlerunning{Mobile Networks: From 4G to 5G}

\author{João Teixeira \and José Ferreira \and Miguel Solino}

\authorrunning{João Teixeira \and José Ferreira \and Miguel Solino}

\institute{
Universidade do Minho, Departamento de  Informática, 4710-057 
Braga, Portugal\\
e-mail: \{a85504,a83683,a86435\}@alunos.uminho.pt
}

\date{}
\bibliographystyle{splncs}

\maketitle
\begin{abstract}
Todas as pessoas sonham puder descarregar ficheiros da Internet a
velocidades alucinantes.
Inevitavelmente, esse sonho acabou por sair das nossas casas e
seguir-nos para a rua à medida que os dispositivos se tornavam mais
compactos e mais potentes.
A solução prontamente apresentada pelas empresas foi uma conexão
móvel cada vez mais rápida.
Eventualmente foi atingida uma barreira na procura incessante da
velocidade, o 4G. A resposta das empresas a este entrave foi o
desenvolvimento de um novo paradigma denominado de 5G.
\end{abstract}

\section{Introdução}

\section{Implementação do 4G}
A fim de melhor entender as vantagens proporcionadas pelo 5G temos
de previamente compreender a norma que este vem substituir.

\subsection{A barreira do 4G}
\section{Implementação do 5G}
\subsection{Problemas resolvidos}
\subsection{Desvantagens inerentes}
\subsection{Utilização}
\section{Projetos 5G}
Conhecendo já do que se trata o 5G podemos ver que as vantagens e 
benefícios que isto trará à população em geral ainda são algumas.
Caso isso fosse mentira não teriamos o exemplo do "projeto" que a
Huawei está a seguir. Neste momento, esta empresa encontra-se numa 
tentativa de expandir esta nova potencialidade pelo mundo todo e,
algo merecido de ser evidenciado, é que se está a focar bastante em
países pouco desenvolvidos, ajudando assim a deixarem de o ser e melhorarem as suas
condições económicas, sendo eles Cabo Verde, Moçambique, Brasil, entre
outros. No caso do ultimo país referido, está previsto um investimento
de pelo menos 800 milhões de dólares na construção de uma fábrica em 
São Paulo. Além de isto ter como objetivo ajudar estes países, não significa 
que seja o único propósito. Uma das principais razões de a Huawei investir na compra
de uma empresa no Brasil é para evitar que a guerra entre a China e os Estados Unidos
afete os negócios da empresa na América. Ou seja, este projeto tem o seu lado benéfico
para a sociedade em geral mas também é para evitar que a empresa perca o poder económico
que tem pelo mundo.
Ao contrário dos Estados Unidos, que se recusa a aceitar a Huawei (e devido a isto
a mesma já começou a desenvolver antenas sem ajuda e peças Americanas), a Rússia recebe de braços
abertos esta tecnologia inovadora. Relativamente a Portugal, já existem empresas a querer
investir no 5G, sendo uma delas a Dense Air Portugal que pretende distribuir esta novidade
pelo país mas afirma aplicar de uma forma diferente. Não querem ser compreendidos como 
uma operadora mas sim trabalharem com as mesmas para distribuir este serviço. Afirma também
que em alguns edificios de Lisboa e Porto já foram realizados mais de 600 mil ensaios e pelo menos 39 mil edificios são capazes de serem melhorados para suportar 5G.
\section{Conclusões}

Neste trabalho...

%UNCOMMENT para a bibliografia 
%% ficheirodebibliografia.bib
%\bibliography{ficheirodebibliografia}

%ou inserir directamente os vários \bibitem 
\begin{thebibliography}{1}
\bibitem{Zadeh65}
Zadeh, L.:
\newblock {Fuzzy sets} (1965)

\bibitem{Nguyen99}
Nguyen, H., Walker, E.:
\newblock {First course in fuzzy logic}.
\newblock {Boca Raton: Chapman and Hall/CRC Press} (1999)
\end{thebibliography}

\end{document}
