\documentclass{llncs}
\usepackage{times}
\usepackage[T1]{fontenc}

\usepackage{a4}
\usepackage[margin=3cm,nohead]{geometry}
\usepackage{epstopdf}
\usepackage{graphicx}
\usepackage{fancyvrb}
\usepackage{amsmath}
\renewcommand{\baselinestretch}{1.5}

\usepackage[utf8]{inputenc}
\usepackage[portuges]{babel}


\begin{document}
\mainmatter
\title{Mobile Networks: From 4G to 5G}

\titlerunning{Mobile Networks: From 4G to 5G}

\author{João Teixeira \and José Ferreira \and Miguel Solino}

\authorrunning{João Teixeira \and José Ferreira \and Miguel Solino}

\institute{
Universidade do Minho, Departamento de  Informática, 4710-057 
Braga, Portugal\\
e-mail: \{a85504,a83683,a86435\}@alunos.uminho.pt
}

\date{}
\bibliographystyle{splncs}

\maketitle
\begin{abstract}
Todas as pessoas sonham puder descarregar ficheiros da Internet a
velocidades alucinantes.
Inevitavelmente, esse sonho acabou por sair das nossas casas e
seguir-nos para a rua à medida que os dispositivos se tornavam mais
compactos e mais potentes.
A solução prontamente apresentada pelas empresas foi uma conexão
móvel cada vez mais rápida.
Eventualmente foi atingida uma barreira na procura incessante da
velocidade, o 4G. A resposta das empresas a este entrave foi o
desenvolvimento de um novo paradigma denominado de 5G.
\end{abstract}

\section{Introdução}

Este artigo irá explorar a evolução dos serviços móveis baseados em 4G
LTE para serviços móveis baseados na nova tecnologia de ponta denominada de 5G assim como as suas vantagens e desvantagens.

\section{Caminho para o 4G}
A fim de melhor entender as vantagens proporcionadas pelo 5G temos
de previamente conhecer a forma como a comunicação móvel tem vindo a
evoluir ao longo do tempo e compreender a norma que esta vem
substituir.
A primeira geração, desenvolvida durante a década de 80, era meramente
analógica permitindo velocidades de comunicação extremamente lentas,
pelos \textit{standards} modernos. Apesar de ser relativamente
rudimentar, como o desenvolvimento deste método foram resolvidos
múltiplos problemas, tais como o multiplexing da banda de frequências
e comunicação não interrompida.
Desde aí várias tecnologias novas foram implementadas.
Com a introdução da segunda geração, na década de 1990, a comunicação
passou a ser digital e foi acrescentado o suporte para mais
utilizadores. Por fim, a terceira geração introduziu maiores
velocidades de transferência.
Por fim, com a quarta geração de comunicação móvel, denominada de 4G
LTE, existiram inúmeros avanços.
Passou a ser possível passar de uma torre para outra 


\section{Implementação do 4G}
\subsection{A barreira do 4G}
\section{Implementação do 5G}
\subsection{Problemas resolvidos}
\subsection{Desvantagens inerentes}
\subsection{Utilização}
\section{Projetos 5G}

\section{Conclusões}

Neste trabalho...

%UNCOMMENT para a bibliografia 
%% ficheirodebibliografia.bib
%\bibliography{ficheirodebibliografia}

%ou inserir directamente os vários \bibitem 
\begin{thebibliography}{1}

\bibitem{Kaleem12}
M.Kaleem Iqbal, M. Bilal Iqbal, I. Rasheed, A. Sandhu:
\newblock{4G Evolution and Multiplexing Techniques with solution to
implementation challenges} (2012)

\bibitem{Upkar12}
Upkar Varshney:
\newblock{4G Wireless Networks} (2012)

\bibitem{Boyd12}
B. Bangerter, S. Talwar, R. Arefi, K. Stewart, Intel:
\newblock{Networks and Devices for the 5G Era} (2014)

\bibitem{Lauridsen17}
M. Lauridsen, L. C. Giménez, I. Rodriguez, T. B. Sørensen, P. Mogensen:
\newblock{From LTE to 5G for Connected Mobility} (2017)

\bibitem{Agiwal16}
M. Agiwal, A. Roy, N. Saxena:
\newblock{Next Generation 5G Wireless Networks: A Comprehensive Survey} (2016)

\end{thebibliography}

\end{document}
