\documentclass[a4paper]{report}
\usepackage[utf8]{inputenc}
\usepackage[portuguese]{babel}
\usepackage{hyperref}
\usepackage{a4wide}
\hypersetup{pdftitle={TP2:  Protocolo IP  (Parte I e II)},
pdfauthor={João Teixeira, José Ferreira, Miguel Solino},
colorlinks=true,
urlcolor=blue,
linkcolor=black}
\usepackage{subcaption}
\usepackage[cache=false]{minted}
\usepackage{listings}
\usepackage{booktabs}
\usepackage{multirow}
\usepackage{appendix}
\usepackage{tikz}
\usepackage{authblk}
\usepackage{bashful}
\usepackage{verbatim}
\usepackage{amsmath}
\usetikzlibrary{positioning,automata,decorations.markings}
\AfterEndEnvironment{figure}{\noindent\ignorespaces}

\begin{document}

\title{TP2:  Protocolo IP (Parte I e II)\\ 
\large Grupo Nº 7}
\author{João Teixeira (A85504) \and José Ferreira (A83683) \and Miguel Solino (A86435)}

\date{\today}

\begin{center}
    \begin{minipage}{0.75\linewidth}
        \centering
        \includegraphics[width=0.4\textwidth]{images/eng.jpeg}\par\vspace{1cm}
        \vspace{1cm}
        \href{https://www.uminho.pt/PT}
        {\color{black}{\scshape\LARGE Universidade do Minho}} \par
        \vspace{1cm}
        \href{https://www.di.uminho.pt/}
        {\color{black}{\scshape\Large Departamento de Informática}} \par
        \maketitle
    \end{minipage}
\end{center}

\tableofcontents

\pagebreak
\chapter{Parte 1}
\section{Exercício 1}

\begin{figure}[H]
    \centering 
    \includegraphics[width=\textwidth]{images/coreEx1.png}  
    \caption{Ex. 1 - Core}
    \label{fig:coreEx1}
\end{figure}

\subsection{Alínea a}
\textbf{Active o wireshark ou o tcpdump no pc s1. Numa shell de s1, execute o
comando traceroute -l para o endereço IP do host h5.}

\begin{figure}[H]
    \centering 
    \includegraphics[width=\textwidth]{images/traceroutEx1.png}  
    \caption{Ex. 1 - Traceroute}
    \label{fig:traceroutEx1}
\end{figure}
\subsection{Alínea b}
\textbf{Registe e analise o tráfego ICMP enviado enviado por s1 e tráfego ICMP
recebido como resposta. Comente os resultados face ao comportamento esperado.}\\
Analisando os resultados obtidos, constatamos que o envio de pacotes teve duas
fases.\\
As fases estão divididas entre os pacotes com TTL abaixo de 4 e os com TTL acima
de 4.\\
Na primeira fase, os pacotes com TTL = 1, TTL = 2 e TTL = 3 foram descartados
pelos routers r1, r2 e r3, respetivamente. Para cada um destes foi recebido um
pacote \textit{Time-to-live exceeded}.\\
Na segunda fase, ao contrário dos pacotes anteriores, nenhum deles foi
descartado tendo como resposta pacotes \textit{Echo (ping) reply}.

\begin{figure}[H]
    \centering 
    \includegraphics[width=\textwidth]{images/wiresharkEx1.png}  
    \caption{Ex. 1 - Wireshark}
    \label{fig:wiresharkEx1}
\end{figure}
\subsection{Alínea c}
\textbf{Qual deve ser o valor inicial mínimo do campo TTL para alcançar o
destino h5? Verifique na prática que a sua resposta está correta.}\\
Face aos resultados analisados na questão anterior, verifica-se que a partir de
TTL = 4 os pacotes deixam de receber mensagem de erro como resposta, logo o
valor inicial mínimo para alcançar o destino h5 será 4.

\subsection{Alínea d}
\textbf{Qual o valor médio do tempo de ida-e-volta (Round-Trip Time) obtido?}\\
O valor médio é obtido calculando a seguinte equação:
\begin{math}
RTT = ((0.075 + 0.013 + 0.010) / 3 + (0.022 + 0.012 + 0.012) / 3 + (0.025 +
0.014 + 0.013) / 3 + (0.041 + 0.016 + 0.015) / 3) * 2 = 0.178
\end{math}

\section{Exercício 2}

\begin{figure}[H]
    \centering 
    \includegraphics[width=\textwidth]{images/wiresharkEx2.png}  
    \caption{Ex. 2 - Wireshark}
    \label{fig:wiresharkEx2}
\end{figure}

\subsection{Alínea a}
\textbf{Qual é o endereço IP da interface ativa do seu computador?}
\begin{figure}[H]
    \centering 
    \includegraphics[width=\textwidth]{images/ipEx2.png}
    \caption{Ex. 2 - Cabeçalho IP}
    \label{fig:ipEx2}
\end{figure}
192.168.100.208

\subsection{Alínea b}
\textbf{Qual é o valor do campo protocolo? O que identifica?}\\
ICMP (1).\\
O valor do campo protocolo é 1, ou seja, identifica o protocolo ICMP.

\subsection{Alínea c}
\textbf{Quantos bytes tem o cabeçalho IP(v4)? Quantos bytes tem o campo de dados
(payload) do datagrama? Como se calcula o tamanho 
do payload?}\\
Analisando o campo Header length na figura \ref{fig:ipEx2}, conclui-se que o
cabeçalho IP tem 20 bytes.\\
O tamanho do campo de dados (\textit{payload}) é a diferença entre o número 
total de bytes e o tamanho do cabeçalho do datagrama. Logo, \textit{Payload} =
60 - 20 = 40 bytes.

\subsection{Alínea d}
\textbf{O datagrama IP foi fragmentado? }

\begin{figure}[H]
    \centering 
    \includegraphics[width=\textwidth]{images/ipEx2Flags.png}
    \caption{Ex. 2 - Flags do Cabeçalho IP}
    \label{fig:ipEx2Flags}
\end{figure}
Observando a figura \ref{fig:ipEx2Flags} reparamos que no campo Flags, tudo está
a 0, logo o \textit{Fragment offset} tem valor 0, o que Tendo em atenção agora a
flag \textit{More Fragments} concluímos que não existem mais fragmentos pois, se
o valor for 1 existem, caso contrário é 0. Logo, se estamos no primeiro
fragmento e não existem mais então este é o datagrama original.

\subsection{Alínea e}
\textbf{Ordene os pacotes capturados de acordo com o endereço IP fonte (e.g.,
selecionando o cabeçalho da coluna Source), e analise a sequência de tráfego
ICMP gerado a partir  do endereço IP atribuído à interface da sua máquina. Para
a sequência de mensagens ICMP enviadas pelo seu computador, indique que campos
do cabeçalho IP variam de pacote para pacote.}\\
Após ordenarmos, concluímos que os campos do cabeçalho IP que variam de pacote
para pacote são o \textit{TTL}, \textit{Header Checksum} e o identificador.

\begin{figure}[H]
    \centering 
    \includegraphics[width=\textwidth]{images/wiresharkSourceEx2.png}
    \caption{Ex. 2 - Tráfego Wireshark ordenado por endereço fonte}
    \label{fig:wiresharkSourceEx2}
\end{figure}
\subsection{Alínea f}
\textbf{Observa algum padrão nos valores do campo de Identificação do datagrama
IP e TTL?}\\
Ao analisar o datagrama IP, verificamos que este conserva os primeiros 8 bits em
todos o casos apresentados e os restantes são incrementados sequencialmente.\\
Também observamos que o TTL é incrementado sequencialmente

\subsection{Alínea g}
\textbf{Ordene o tráfego capturado por endereço destino e encontre a série de
repostas ICMP TTL exceeded enviadas ao seu computador. Qual é o valor do campo
TTL? Esse valor permanece constante para todas as mensagens de resposta ICMP TTL
exceed enviados ao seu host? Porquê?}\\
O valor do campo TTL é 62.\\
Para todas as mensagens de resposta \textit{ICMP TTL exceeded} recebidas no
nosso host esse valor manteve-se constante.\\
Apear do TTL observado (pré-definido pelo destino), quando o pacote chega ao
destino o TTL é de 61. Tal devesse ao facto do pacote em questão ter passado por
2 routers antes de chegar e, por isso, ter sido decrementado 2 vezes.

\begin{figure}[H]
    \centering 
    \includegraphics[width=\textwidth]{images/wiresharkDestinyEx2.png}
    \caption{Ex. 2 - Tráfego Wireshark ordenado por endereço de destino}
    \label{fig:wiresharkDestinyEx2}
\end{figure}

\section{Exercício 3}

\begin{figure}[H]
    \centering 
    \includegraphics[width=\textwidth]{images/datagramaIpEx3.png}
    \caption{Ex.3 - Fragmentos do datagrama IP}
    \label{fig:datagramaIpEx3}
\end{figure}
\subsection{Alínea a}
\textbf{Localize a primeira mensagem ICMP. Porque é que houve necessidade de
fragmentar o pacote inicial?}\\
Analisando a figura, vemos que a primeira mensagem ICMP é a 214.\\
Visto que o tamanho permitido pelo protocolo é inferior ao tamanho do PDU é
necessário que o pacote inicial seja fragmentado para poder circular na rede.

\subsection{Alínea b}
\textbf{Imprima o primeiro fragmento do datagrama IP segmentado. Que informação
no cabeçalho IP indica que se trata do primeiro fragmento? Qual é o tamanho
deste datagrama IP?}

\begin{figure}[H]
    \centering 
    \includegraphics[width=\textwidth]{images/fragmentDatagramaIpEx3.png}
    \caption{Ex.3 - Cabeçalho do primeiro fragmento do datagrama IP}
    \label{fig:fragmentDatagramaIpEx3}
\end{figure}
Podemos observar na figura que, no campo Flags, o \textit{More fragments} tem
valor 1, isso indica que o diagrama foi fragmentado, existindo então mais
fragmentos.\\
Seguidamente, podemos ver que o \textit{Fragment offset} é 0, provando de que se
trata do primeiro fragmento.\\
A \textit{Total Length} e igual a 1500 bytes

\subsection{Alínea c}
\textbf{Imprima o segundo fragmento do datagrama IP original. Que informação do
cabeçalho IP indica que não se trata do 1º fragmento? Há mais fragmentos? O que
nos permite afirmar isso?}

\begin{figure}[H]
    \centering 
    \includegraphics[width=\textwidth]{images/fragment2DatagramaIpEx3.png}
    \caption{Ex.3 - Cabeçalho do segundo fragmento do datagrama IP}
    \label{fig:fragment2DatagramaIpEx3}
\end{figure}
Tal como já foi referido anteriormente, para se verificar se um fragmento é o
primeiro basta ter em atenção o valor que está no \textit{Fragment offset}. Se
esse valor for 0 então podemos concluir que se trata do primeiro. Como o valor
apresentado é diferente de 0 pudemos concluir que o fragmento em questão não se
trata do primeiro.\\
Podemos concluir que existem mais fragmentos pois o bit correspondente ao
\textbf{More fragments} é igual a 1.

\subsection{Alínea d}
\textbf{Quantos fragmentos foram criados a partir do datagrama original?
Como se detecta o último fragmento correspondente ao datagrama original?}\\
Como está mostrado na imagem, o terceiro fragmento do datagrama original
tem o bit correspondente a \textit{More fragments} a 0, ou seja, não há mais 
fragmentos a seguir a este. Concluindo assim que foram criados 3 fragmentos 
(214, 215 e 216).

\subsection{Alínea e}
\textbf{Indique, resumindo, os campos que mudam no cabeçalho IP entre os
diferentes fragmentos, e explique a forma como essa informação permite
reconstruir o datagrama original.}\\
Ao longo dos diferentes fragmentos, os campos do \textit{Fragment offset} e do
\textit{More Fragments} são alterados no cabeçalho IP.\\
O primeiro permite identificar a posição do fragmento no datagrama original. O
segundo indica se existem mais fragmentos do datagrama original para além do
próprio.


\chapter{Parte 2}

\section{Exercício 1}

\subsection{Alínea a}
\textbf{Indique que endereços IP e máscaras de rede foram atribuídos pelo CORE a
cada equipamento. Para simplificar, pode incluir uma imagem que ilustre de forma
clara a topologia definida e o endereçamento usado.}

\begin{figure}[H]
    \centering 
    \includegraphics[width=\textwidth]{images/topologiaCore.png}
    \caption{Topologia Core}
    \label{fig:topologiaCore}
\end{figure}
Na figura \ref{fig:topologiaCore}, é possível verificar os endereços atribuídos
a cada equipamento. 

\subsection{Alínea b}
\textbf{Trata-se de endereços públicos ou privados? Porquê?}\\
Uma vez que todos os endereços utilizam um dos blocos reservados a endereços
privados: "10.0.0.0 - 10.255.255.255 / 8", concluimos que se tratam de endereços
privados.

\subsection{Alínea c}
\textbf{Por que razão não é atribuído um endereço IP aos switches?}\\
Não é necessário a atribuição de endereços IP aos switches porque são 
intervenientes na camade de ligação 2 e por sua vez transparentes à camada de
ligação 3. Estes encaminham os pacotes apenas tendo em atenção os endereços MAC
dos equipamentos.\\

\subsection{Alínea d}
\textbf{Usando o comando ping certifique-se que existe conectividade IP entre 
os laptops dos vários departamentos e o servidor do departamento A 
(basta certificar-se da conectividade de um laptop por departamento).}\\

\begin{figure}[H]
    \centering 
    \includegraphics[width=\textwidth]{images/pingEx1P2.png}
    \caption{Ex.1 - Ping}
    \label{fig:pingEx1P2}
\end{figure}
Como podemos observar pela figura \ref{fig:pingEx1P2}, para verificar se existia
conectividade foi utilizado o comando ping em pelo menos um laptop de cada
departamento. Sendo que todos obtiveram resposta do servidor após terem enviado
pacotes, concluímos que existe conectividade em todos os departamentos.

\subsection{Alínea e}
\textbf{Verifique se existe conectividade IP do router de acesso Rext para o
servidor S1.}\\

\begin{figure}[H]
    \centering 
    \includegraphics[width=\textwidth]{images/RextPing.png}
    \caption{Ex.2 - Conectividade S1 - Router Rext}
    \label{fig:RextPing}
\end{figure}
Observando a figura \ref{fig:RextPing} e seguindo o racíocinio da alinea
anterior, verificamos que existe conectividade do router Rext para o servidor
S1.

\section{Exercício 2}

\subsection{Alínea a}
\textbf{Execute o comando netstat –rn por forma a poder consultar a tabela de
encaminhamento unicast (IPv4). Inclua no seu relatório as tabelas de
encaminhamento obtidas; interprete as várias entradas de cada tabela. Se
necessário, consulte o manual respetivo (man netstat).}

\begin{figure}[H]
    \centering 
    \includegraphics[width=\textwidth]{images/netstatPcEx2P2.png}
    \caption{Ex.2 - netstat pc departamento B}
    \label{fig:netstatPcEx2P2}
\end{figure}

\begin{figure}[H]
    \centering 
    \includegraphics[width=\textwidth]{images/netstatRouterEx2P2.png}
    \caption{Ex.2 - netstat Router departamento B}
    \label{fig:netstatRouterEx2P2}
\end{figure}
Na figura \ref{fig:netstatPcEx2P2} está representada a tabela de encaminhamento
de onde podemos retirar informações relativa à rota que irá ser feita pelo
pacote. Na coluna "Destination" é nos indicado a sub-rede destino, na "Gateway"
a informação do equipamento pelo qual irá passar o pacote e na "Genmask" o tipo
da máscara.\\
Analisando a primeira e a segunda entrada da figura \ref{fig:netstatPcEx2P2} 
<<<<<<< HEAD
reparamos que as duas diferem no Gateway. Sendo que os dois endereços estão ligados
diretamente entre si, o Gateway não precisa de ser definido. Caso contrário, seria
necessário para se saber o próximo salto.\\
Relativamente à coluna Flags, estas apenas servem para acrescentar informações adicionais.
A flag "UG" é utilizada quando o gateway está definido e a "U" caso contrário.\\
Olhando agora apenas para a primeira entrada, reparamos que a máscara é 0 o que 
significa que o destino 0.0.0.0 indentifica todas as redes possíveis.\\
Sendo que a primeira tabela e a segunda tabela são idênticas, não é necessário fazer
uma análise de todas as entradas da segunda.

=======
reparamos que as duas diferem no Gateway. Sendo que os dois endereços estão
ligados diretamente entre si, o Gateway não precisa de ser definido. Caso
contrário, seria necessário para se saber o próximo salto.\\
Relativamente à coluna Flags, estas apenas servem para acrescentar informações
adicionais. A flag "UG" é utilizada quando o gateway está definido e a "U" caso
contrário. Sendo que a primeira tabela e a segunda tabela são idênticas, não é
necessário fazer uma análise de todas as entradas da segunda.
>>>>>>> 3c7872edbc070c9da5e666ddb0d33c1d70118dbe

\subsection{Alínea b}
\textbf{Diga, justificando, se está a ser usado encaminhamento estático ou
dinâmico (sugestão: analise que processos estão a correr em cada sistema).}

\begin{figure}[H]
    \centering 
    \includegraphics[width=\textwidth]{images/htop2a.png}
    \caption{Ex.2 - Processos a serem executados em Ra}
    \label{fig:htop2a}
\end{figure}

\begin{figure}[H]
    \centering 
    \includegraphics[width=\textwidth]{images/htopca1.png}
    \caption{Ex.2 - Processos a serem executados em Ca1}
    \label{fig:htopca1}
\end{figure}
Se apenas analisassemos a figura \ref{fig:htop2a} reparavamos que é usado pelo router
o protocolo ospfd (este permite que o pacote siga diferentes caminhos quando um não é
possível) e assim concluiamos que se trata de um encaminhamento dinâmico. No entanto,
após analisar a figura \ref{fig:htopca1} verificamos que não é usado esse protocolo,
conluindo então que se trata de um encaminhamento estático.

\subsection{Alínea c}
\textbf{Admita que, por questões administrativas, a rota por defeito (0.0.0.0 ou
default) deve ser retirada definitivamente da tabela de encaminhamento do
servidor S1 localizado no departamento A. Use o comando route delete para o
efeito. Que implicação tem esta medida para os utilizadores da empresa que
acedem ao servidor? Justifique.}

\begin{figure}[H]
    \centering 
    \includegraphics[width=\textwidth]{images/routeDelete.png}
    \caption{Ex.2 - Remoção da rota default}
    \label{fig:routeDelete}
\end{figure}

\begin{figure}[H]
    \centering 
    \includegraphics[width=\textwidth]{images/pingRextS1.png}
    \caption{Ex.2 - Ping do Rext para S1}
    \label{fig:pingRextS1}
\end{figure}

\begin{figure}[H]
    \centering 
    \includegraphics[width=\textwidth]{images/pingS1Rext.png}
    \caption{Ex.2 - Ping do S1 para Rext}
    \label{fig:pingS1Rext}
\end{figure}

Removendo a rota por defeito é perdida a conectividade entre o servidor S1 e os 
restantes hosts existentes fora do departamento onde se encontra. Isto acontece porque o 
servidor S1 não tendo definida a rota de envio de tráfego para redes não locais faz com 
que não saiba para onde enviar de volta o que recebeu dos utilizadores. Tal verifica-se
nas figuras \ref{fig:pingRextS1} e \ref{fig:pingS1Rext} em que o router consegue enviar
para o servidor pacotes mas não os recebe de volta e o servidor não consegue enviar
pacotes para o router.

\subsection{Alínea d}
\textbf{Adicione as rotas estáticas necessárias para restaurar a conectividade
para o servidor S1 por forma a contornar a restrição imposta na alínea c).
Utilize para o efeito o comando route add e registe os comandos que usou.}

\begin{figure}[H]
    \centering 
    \includegraphics[width=\textwidth]{images/routeAdd.png}
    \caption{Ex.2 - Restauração da conectividade adicionando rotas estáticas}
    \label{fig:routeAdd}
\end{figure}

\subsection{Alínea e}
\textbf{Teste a nova política de encaminhamento garantindo que o servidor está
novamente acessível utilizando para o efeito o comando ping. Registe a nova
tabela de encaminhamento do servidor.}

\begin{figure}[H]
    \centering 
    \includegraphics[width=\textwidth]{images/conectividadeDep.png}
    \caption{Ex.2 - Conectividade entre os departamentos e S1}
    \label{fig:conectividadeDep}
\end{figure}

\begin{figure}[H]
    \centering 
    \includegraphics[width=\textwidth]{images/tabS1.png}
    \caption{Ex.2 - Tabela de encaminhamento}
    \label{fig:tabS1}
\end{figure}

Analisando as figuras \ref{fig:conectividadeDep} e \ref{tabS1} podemos verificar que
existe conectividade entre o servidor S1 e os laptops de todos os departamentos.

\section{Exercício 3}

\subsection{Alínea 1}

\begin{figure}[H]
    \centering 
    \includegraphics[width=\textwidth]{images/topCoreEx3.png}
    \caption{Ex.2 - Topologia do novo endereçamento}
    \label{fig:topCoreEx3}
\end{figure}

\textbf{Considere que dispõe apenas do endereço de rede IP 172.yyx.32.0/20, em
que “yy” são os dígitos correspondendo ao seu número de grupo (Gyy) e “x” é o
dígito correspondente ao seu turno prático (PLx). Defina um novo esquema de
endereçamento para as redes dos departamentos (mantendo a rede de acesso e core
inalteradas) e atribua endereços às interfaces dos vários sistemas envolvidos.
Deve justificar as opções usadas.}

\subsection{Alínea 2}
\textbf{Qual a máscara de rede que usou (em notação decimal)? Quantos interfaces
IP pode interligar em cada departamento? Justifique.}

\subsection{Alínea 3}
\textbf{Garanta e verifique que a conectividade IP entre as várias redes locais
da organização MIEI-RC é mantida. Explique como procedeu.}

\chapter{Conclusão}
Com este trabalho consideramos que conseguimos atingir com sucesso todos os
desafios apresentados no enunciado obtendo assim um conhecimento mais avançado
sobre o protocolo IPv4 e \textit{sub-netting}.

\end{document}
