\documentclass[a4paper]{report}
\usepackage[utf8]{inputenc}
\usepackage[portuguese]{babel}
\usepackage{hyperref}
\usepackage{a4wide}
\hypersetup{pdftitle={TP3:  Protocolo IP},
pdfauthor={João Teixeira, José Ferreira, Miguel Solino},
colorlinks=true,
urlcolor=blue,
linkcolor=black}
\usepackage{subcaption}
\usepackage[cache=false]{minted}
\usepackage{listings}
\usepackage{booktabs}
\usepackage{multirow}
\usepackage{appendix}
\usepackage{tikz}
\usepackage{authblk}
\usepackage{bashful}
\usepackage{verbatim}
\usepackage{amsmath}
\usetikzlibrary{positioning,automata,decorations.markings}
\AfterEndEnvironment{figure}{\noindent\ignorespaces}

\begin{document}

\title{TP3:\\ 
\large Grupo Nº 7}
\author{João Teixeira (A85504) \and José Ferreira (A83683) \and Miguel Solino (A86435)}

\date{\today}

\begin{center}
    \begin{minipage}{0.75\linewidth}
        \centering
        \includegraphics[width=0.4\textwidth]{images/eng.jpeg}\par\vspace{1cm}
        \vspace{1cm}
        \href{https://www.uminho.pt/PT}
        {\color{black}{\scshape\LARGE Universidade do Minho}} \par
        \vspace{1cm}
        \href{https://www.di.uminho.pt/}
        {\color{black}{\scshape\Large Departamento de Informática}} \par
        \maketitle
    \end{minipage}
\end{center}

\tableofcontents

\pagebreak
\chapter{Parte 1}
\section{Exercício 1}
\subsection{Alínea a}
\textbf{Considere que dispõe apenas do endereço de rede IP 172.yyx.32.0/20, em
que “yy” são os dígitos correspondendo ao seu número de grupo (Gyy) e “x” é o
dígito correspondente ao seu turno prático (PLx). Defina um novo esquema de
endereçamento para as redes dos departamentos (mantendo a rede de acesso e core
inalteradas) e atribua endereços às interfaces dos vários sistemas envolvidos.
Deve justificar as opções usadas.}

\chapter{Conclusão}
Neste trabalho prático foi nos dada a oportunidade de enriquecer o nosso conhecimento
relativamente a redes. Para isso foram usadas duas ferramentas: Core, para simulação de 
redes, e Wireshark, para captura de tráfego.\\
O trabalho está dividido em duas partes e cada uma delas dividida em 3 questões.
Cada parte teve como foco um tema e o mesmo acontece para cada questão facilitando a
nossa perceção em detalhes que mesmo não parecendo são relevantes.\\
Na primeira parte, o objetivo principal foi analisar o IP (Internet Protocol) e para isso
analisamos o formato dos pacotes/datagramas IP e fragmentação dos mesmos.\\
A segunda parte tinha como foco o estudo e perceção do endereçamento e encaminhamento do
que foi estudado na parte anterior. Construimos topologias na tentativa de simular 
o máximo do que se passa na realidade e nas mesmas remover, alterar e adicionar 
endereços para compreender como são feitos os transportes de pacotes numa rede.\\
Resumindo, consideramos que conseguimos atingir com sucesso todos os
desafios apresentados no enunciado obtendo assim um conhecimento mais avançado
sobre o protocolo IPv4 e \textit{sub-netting}.

\end{document}
